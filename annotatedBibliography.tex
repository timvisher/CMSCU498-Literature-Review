\documentclass[12pt,oneside,letterpaper,titlepage]{article}
\author{Tim Visher\\ CMSCU 310\\ Chestnut Hill College}
\title{Annotated Bibliography}
\date{November 20, 2009}

\begin{document}

\maketitle

\section{Web 2.0}

Alexander, B. (2006). Web 2.0: New Wave of Innovation for Teaching and Learning?. EDUCAUSE Review, 41(2), 32-44

This author also starts out attempting to explain what Web 2.0 is.  He
identifies several major themes, such as openness, collaboration, and
folksonomies and uses them to draw the distinguishing line between Web 1.0 and
Web 2.0.  The claim is that it is these ideas and attitudes that are the real
game changers.  He then goes on to discuss the applicability of Web 2.0 in the
context of higher learning institutions.  Of special interest to him seems to be
the rich pedagogical possibilities inherent in these collaborative
techonologies.  He highlights especially social bookmarking as a way to draw out
important themes that the communities tagging those events hold.  Finally, the
author spends some time outlining where he believes Web 2.0 is going, predicting
that unlike the dot com bubble of the turn of the millenium, this trend is here
to stay.

This article eloquently summarizes thet spirit of Web 2.0 technology.  It should
be useful regarding the discussion of the 'why' of Web 2.0 as well as its
possible uses in the 'real world'.

Barsky, E., Purdon, M. (2006). Introducing Web 2.0: social networking and social bookmarking for health librarians. JCHLA, 27(3), 65-7.

The authors here provide a brief overview of what Web 2.0 is and what it is
about.  They state that it is primarily about participation and information,
rather than about commerce (as they claim Web 1.0 was typified by).  They focus
specifally on two areas of Web 2.0 and how they pertain to Librarians in the
Health Industry.  Social Networks (Facebook, MySpace, and the like), they claim,
can be used to facilitate the sharing of information between both medical
practicioners and clientelle.  The ability to quickly construct focussed
micro-communities is key to this.  Also, they note Social Bookmarking services
(Delicious, Furl) that have cropped up.  These services are commonly used to
create folksonomies but can just as easily be used by professional librarians to
create Internet guides for use by their clients.

This article documents another way that the medical industry specifically is
using Web 2.0 technologies as well as giving more language to what Web 2.0 is.
It should be useful both in documenting and describing Web 2.0 as well as
showing how it is being used in professional fields.

Giustini, D. (2006). How Web 2.0 is changing medicine: Is a medical Wikipedia the next step?. BMJ, 333(7582), 1283-4.

In this editorial, the author gives a short account of what Web 2.0 is,
emphasizing that Web 2.0's goals chiefly appear to be the use of the Internet to
share information and collaborate with each other.  Also of note would be is an
emphasis on the use of free services delivered over the Web as a Platform to
accomplish this sharing and collaborating.  The author has an emphasis on the
use of Web 2.0 technology in the medical industry.

He documents one case of a blog that has been quite successful in the medical
community, citing that the author's consistent sharing of timely information and
the rich community that has grown up around it that comments on and discusses
the information as the reason for its success.  Another distinctly Web 2.0
technology, RSS, is also cited in the article as beginning to change the way
medical practitioners work.  In a the busy world in which we live, doctors
simply don't have time to keep up with all of the disparate publications that
come to their door.  With RSS, they have a consistent interface through which
all the information flows.  Finally, he discusses the possibility of using Wiki
technology in the medical community.

This is an interesting editorial which does a good job not only of capturing the
essence of what Web 2.0 is about, but also documenting how it is already
beginning to change the medical community at large.  I will be using this in the
Web 2.0 section of my literature review when talking about the use of the Web as
Platform.

Miller, P. (2005). Web 2.0: Building the New Library. Ariadne. 45. Retrieved 15 December, 2008 from http://www.ariadne.ac.uk/issue45/miller/

The author here outlines in great detail the core values and principles of Web
2.0.  Afterwards, he briefly discusses the applicability of this new paradigm in
the context of libraries.

The core values that the author enumerates are really quite detailed and good.
When I'm defining Web 2.0, this article should be very useful.

O'Reilly, T. (2007). What is Web 2.0: Design Patterns and Business Models for the Next Generation of Software. Communications and Strategies, No. 1, p. 17. Retrieved 19 September, 2008, from http://papers.ssrn.com/sol3/papers.cfm?abstract_id=1008839

O'Reilly's seminal paper on the advent of Web 2.0 technology and its
distinguishing characteristics is recognized in the field as an authoritative
work of definition and forethought in the area of Cloud Computing or software as
a service.  This should be very useful to my research as it is well recognized
and oft-cited by others.

\section{Thin Client Architecture}

Kanter, J. (1998) Understanding Thin-Client/Server Computing. Microsoft Strategic Technology Series. Redmond, Washington: Microsoft Press

The author here provides an in-depth overview of the thin-client computing
model.  He covers what thin-client computing is, how it works, as well as
several use cases for environments where thin-client computing is particularly
applicable.  In the end, he makes a strong business case for moving to
thin-client computing, as he shows that it significantly reduces TOC.

This article should be useful on multiple levels.  First, it goes in depth into
the technological challenges facing thin-client computing, which should be
useful in attempting to design a netbook.  Secondly, he attempts to build a
business case for using thin-client computing, which should be useful in showing
where netbooks would be applicable business wise.

Lai, E. (2007). Thin Client's Get Microsoft's Approval-For Some Users. Computer World, 41(15), 1,16.

This brief article covers Microsoft's May thin-client turnabout.  Previously,
Microsoft was anti thin-client uses of Windows, apparently due to licensing
issues.  Now, however, they must have worked out their business model and they
are ready to begin offering Thin Client licenses for Vista, but only or a
substantially increased price.

This article should be marginally useful showing the business adoption of the
Thin-Client model.

Lai, A., Nieh, J. (2002) Limits of Wide-Area Thin-Client Computing. Proceedings of the ACM Sigmetrics, 30(1), 228-239.

This article attempts to test the effectiveness of 6 popular thin-client
implementations from a variety of providers in order to understand how
thin-client architecture can be most effectively optimized for WANs.  The
interest is primarily in understanding how thin-clients can be optimized to best
utilize ASPs in the coming Cloud Computing period.

It begins by outlining the experimental design that they would utilize during
their tests.  Since no research had been done prior to this on the effectiveness
of thin-client offerings in the context of WANs, the actual design of the
experimental method is one of the deliverables of this paper.  Because most of
the systems they tested were closed source, they had to verify that their
testing methodology was basically correct.  They did this through testing the
VNC thin-client technology which is a well-understood, open source thin-client
architecture.

The experimental set up consisted of a number of machines located on the two
coasts to ensure that a true WAN experience was being tested.  To have a control
group of sorts, they also used a local isolated testing suite that involved the
use of a network simulator to mimic the East and West (the logical identifiers
of the two testing sites) sites but in a much more controlled way.  They used
three different benchmarks in order to measure the usability of thin-clients:
Latency in which a variety of simple operations were performed to understand the
basic latency of the response times over the thin-client software, Web in which
they used a version of the Ziff-Davis i-Bench web benchmark to load and display
a large number of web pages (108) in succession, and Video in which they played
a 30 second video clip through the thin-client software.

The article concludes that there is a large variety of performance among the
popular thin-client offerings.  They offer suggestions about what can be done to
maximize performance in a WAN context such as optimizing for latency rather than
bandwidth (the opposite of what many systems are doing today).

This article goes into a great amount of detail about what makes thin-client
architecture tick.  It should be very useful in the design of thin-client
systems and the enumeration of what technological advances would have to happen
in order to make thin-client computing in a mobile environment tolerable and
useful.

Richardson, T., Stafford-Fraser, Q., Wood, K. R., Hopper, A. (1998). Virtual Network Computing. IEEE Internet Computing, 2(1), 33-38.

This article presents a history of the VNC (virtual network computing)
technology, which is a thin-client technology allowing people to utilize
computer resources through a thin virtualized client to a machine that is
serving it's interface data out.  It goes on to document at a high level the
parts of the technology that make up VNC and roughly how it is implemented in
order to get the performance that they do.

The main advantage of this model is that the customer can utilize the computing
resources from anywhere without having to carry around any hardware.  Nothing at
all is stored on the client side so everything, including the cursor position
and all, remain intact between accessing sessions.  Also, because of the
technology level at which VNC is implemented (the frame buffer), it works
transparently on any system that has a GUI.

Finally, the document, briefly, future work that they are hoping to do,
including providing implementations for many more device types like Televisions.

I found this article to be a very good overview of the technological reasons
behind the design of VNC.  It should play very nicely in trying to see how an
ultra-mobile netbook could be designed.

Schmidt, B. K., Lam, M. S., Northcutt, J. D. (1999). The interactive performance of SLIM: a stateless, thin-client architecture. Operating Systems Review, 34(5), 32-47.

The authors here collaborate with the product development team from Sun that
developed the Sun Ray(tm) 1 to evaluate the performance of SLIM architecture
products using a newly developed system interaction testing methodology.  The
methodology is actually newly developed because not much testing has been done
in this space before.  They found that utilizing SLIM architecture, they were
able to get performance out of SLIM consoles that was indistinguishable from
more traditional Desktop PCs.

They go on to describe SLIM architecture, which consists of: an Interconnection
Fabric which does not have to be high end, the SLIM protocol which relies on
sending only raw pixel values to the console, thus reducing the need for any
computational power at all on the client, and SLIM Servers which simply
multiplex out the Output and multiplex in the Input for SLIM clients to a
machine running any number of OSes, thus removing the need to implement special
Network Display protocols such as X or ICA.

The authors go into great detail describing their testing methodologies, their
classification of different user profiles, and then their results.  They
conclude that utilizing SLIM architecture is well within the performance limits
of today's common networks.  They key is in their use of a low level protocol
that requires only encoded information about pixels to be sent over the network,
allowing all processing to be done on the server and reducing the intelligence
requirement of the client to nothing more than a frame buffer.

This article is useful mainly because it goes into great detail about what makes
thin clients usable and the limitations on what you can expect a client to do in
such situations.  This is applicable because my eventual goal is to attempt to
design a prototype system that is built to take advantage of ubiquitous
broadband Internet access and Cloud Computing services.

Tynan, D. (2005). Think Thin. InfoWorld, 27(29), 32-36.

The author here does an excellent job documenting the uses of Thin Client
hardware to greatly simplify IT support and provide tremendous functionality to
many different niche industries such as the Medical Industry. It is primarily
concerned with desktop thin clients, however it is very interesting that it
strongly projects the development of a mobile thin client in the near future
that will quickly become the most common computing device in the world. It
covers the balance between power and utility that must be struck in a thin
client.

This is truly an excellent article which should be very useful in documenting
how desktop thin clients are currently being used. It doesn't go into great
technical detail as to how they are designed, but does give the business reasons
for their uses. It has an inset about extending the thin client model to a
mobile environment, which should be useful in my efforts to document the
necessary configuration of a mobile thin-client.

\section{Battery Research}

Brooke, L. (2006, October). CPI takes new direction on Li-ion batteries. Automotive Engineering International, 16-17.

This is an article describing CPI's (a subsidiary of LG Chemical) research
regarding Lithium-Ion secondary battery technology for use in Hybrid Electric
Vehicles.  The key fact here is the use of a manganese cathode instead of the
more traditional LiCoO2 cathode.  The use of this material allows them to
utilize a revolutionary electrode configuration they call the 'stack and fold'
design.  This design allows them to increase the overall calendar life of the
battery, improve on the safety of the technology, and achive greater energy
density and output.

This article will be used in the brief discussion of cathode-anode developments
in the battery research section of my paper.

Ellis, B. l., Makahnouk, W. R. M., Makimura, Y., Toghill, K., Nazar, L. F. (2007). A multifunctional 3.5V iron-based phosphate cathode for rechargeable batteries.  Nature Materials, 6(10), 749-753.

Paper documenting the development of a new cathode material to replace LiFePO4,
A2FePO4F.  It boasts longer life expectancy and higher energy density than the
traditional cathode.

Goodenough, J. B., Padhi, A., Nanjundaswamy, K. S., Masquelier, C. (1997). European Patent No. 20040022447. Free Patents Online: http://www.freepatentsonline.com/EP1501137A2.html

This documents the patent filed by Goodenough et al regarding the development of
LiFePO4 as a Cathode material.  This is one of the materials being used in
today's Lithium Ion batteries.

Hp. (2008). Hp Breaks the 24-Hour Battery Life Barrier. Retrieved November 7, 2008, from http://www.hp.com/sbso/solutions/pc_expertise/battery/?jumpid=em_di_426632_US_US_0_000&diaid=di_hpc_us_720035_US&dimid=1002814586&dicid=taw_Nov08&mrm=1-4BVUP

This news item from HP outlines their new Ultra Capacity Lithium Ion Battery
which they claim can get up to 24 hours of life out of a single charge with very
specific configuration options.

This article should only be marginally useful as it merely outlines the use of
the current battery technology to make today's mobile hardware designs last
longer.

Idota, Y., Kubota, T., Matsufuji, A., Maekawa, Y., Miyasaka, T. (1997). Tin-Based Amorphous Oxide: A High-Capacity Lithium-Ion-Storage Material. Science Magazine, 276(5317), 1395-7.

This article documents the development of a new lithium-ion storage material
that has a higher capacity due to its chemical structure.

Nam, K. T., Kim, D., Yoo, P., Chiang, C., Meethong, N., Hammond, P., Chiang, Y., Belcher, A. (2006). Virus-Enabled Synthesis and Assembly of Nanowires for Lithium Ion Battery Electrodes. Science, 312(5775), 885-8.

Article published to announce the new method developed at MIT to produce
nano-sized wires using viruses.

Samar, B. (1981). U.S. Patent No. 4304825A. European Patent Office: http://v3.espacenet.com/publicationDetails/biblio?CC=US&NR=4304825A&KC=A&FT=D&date=19811208

This is the patent filed by Bell Labs regarding the development of the Graphite
anode and Titanium Sulfide as the cathode.  It was the first patent of a
feasible lithium-ion battery because it used Graphite rather than Lithium Metal
as the anode.  Using Lithium Metal as the anode yields a highly unstable and
dangerous battery.  I plan to use this as a reference in the battery section of
my paper describing the different battery technology developments.

Stanford University (2007, December 20). New Nanowire Battery Holds 10 Times The Charge Of Existing Ones. ScienceDaily. Retrieved January 13, 2009, from http://www.sciencedaily.com /releases/2007/12/071219103105.htm

Documents the successful creation of a nano-wire battery by scientists at
Stanford University.  They claim that the battery has 10 times the energy
density of today's Lithium-Ion batteries, which is a very significant
improvement.

Whittingham, M. S. (1976). Electrical Energy Storage and Intercalation Chemistry. Science Magazine, 192, 1126-1127.

This is Whittingham's publishing of his discovery of the use of Lithium-Ion
technology in rechargeable power sources.  It will be used to provide a
reference for the history section of the battery research.

\section{Time-sharing and Client/Server}

Campbell-Kelly, M., & Aspray, W. (2004). Computer: A History of the Information Machine. Boulder, CO: Westview Press.

This book is another history, more formal than Hackers, of the Computer from its
inceptions (in theory with people like Babbage) all the way through contemporary
times. It also has a significant portion dedicated to the Time Sharing model
which covers the rise and fall of computation as utility.

I hope to use this source to help define and explain Time Sharing as a model and
explain why it fell out of favor. Understanding why Time Sharing fell out of
favor should also help describe why it's coming back in today in the form of
Cloud Computing.

Greenberger, M. (1964). The Computers of Tomorrow. The Atlantic, 213(5), 63-67.

This article has a seminal place in the annals of computing history.  It
describes in some detail the design and applicability of the computer utility
model.  It's written in a bit of a futurist style, but it should still be useful
as it is widely cited as solidifying the concept of the computer utility.

Levy, S. (1994). Hackers: Heroes of the Computer Revolution. New York, NY: Penguin Books.

The author in this book attempts to document the 'other' side of the computer
revolution, which was the Hacker culture that developed at MIT and went on
through loose organizations like the Home Brew Computer Club etc. to help the PC
revolution happen. He covers the history of the computer revolution from the
Tech Model Railroad Company (where the term Hacker was first coined) through to
contemporary culture and theorizing about the future. The primary parts that I
am interested in are the parts on developments in Time-Sharing. He covers the
various technologies that were developed during Time-Sharing, along with why,
and in small ways how.

This book should be very useful in analyzing how Time-Sharing was developed and
why. The stories are mostly anecdotal, but Levy's sources are mostly primary so
they should be fairly trustworthy. It's possible that I could use some of the
information regarding the Client configurations (basically dumb terminals) in
reference to developing a model for today's clients, but it seems unlikely.

Ritchie, D. M., Thompson, K. (1974). The UNIX Time-Sharing System. Communications of the ACM, 17(7), 365-375.

This document is the original description of the first Unix system
implementation created by Ritchie and Thompson.  It contains a full disclosure
of the most important implementation details along with why the implementation
was chosen.  I intend to use it in my discussion of Time-Sharing Systems.

\section{Cloud Computing}

Astley, M., & Bhola, S., & Ward, M. J., & Shagin, K., & Paz, H., & Gershinsky, C. (2008). Pulsar: A resource-control architecture for time-critical service-oriented applications. IBM Systems Journal, 47(2), 265-280.

This article presents a system which is meant to ensure latency targets are met
for a given cloud system. It is intended as a support technology for something
that its authors saw as a critical deficiency in today's technologies which is
that there is currently no way to specify latency targets and ensure that they
are met. This is seen as a critical error especially in the area of Real Time
Systems which need to provide verifiable performance before they are
adopted. They present the case for the need of such a system, the model under
which they developed their system, and a case-study which they use to help
evaluate the effectiveness of Pulsar.

My intentions for the use of this article are simply to utilize their
information on the needs of the server-side hardware that they provide to
extrapolate information about how powerful the specifications for the
client-hardware would need to be. Other than that, most of the information is
not all that useful to me. My topic does not touch on the application
programming models they address nor the actual solution they are attempting to
provide. This article is really only minimally useful.

Baker, S. (2007). Google and the Wisdom of Clouds. Retrieved 13 January, 2009, from http://www.businessweek.com/print/magazine/content/07_52/b4064048925836.htm.

This article documents Christophe Bisiglia's development of Google's cloud
services, from their start as Google 101 to today's offerings as a tool for
researchers and industries the world over.

Bleicher, P. (2006). Solutions Delivered, Not Installed. Applied Clinical Trials, 15(6), 41-44.

The author covers in a fair amount of detail here the possible business models
associated with companies who are or who want to be providing Software as a
Service (or Saas), otherwise known as Cloud Computing solutions. He uses Postini
as an example of a company that is doing very well as a provider of Cloud
Computing services (in their case, Spam filtering), noting how their ability to
constantly upgrade their system as well as configure it for new clients in a
matter of days with no infrastructure required at the client's site as their key
to success. He points out various reasons you might want to move your
corporation's solutions to the cloud: Cost can be driven down as you no longer
have to pay for the infrastructure and on-site technical expertise to maintain
the software, you are no longer responsible for the fine tuning of the software
or its development (which presumably is not your area of expertise anyway), and
you save time by reducing the cost of selecting and then implementing a
solution. He also gives three points to consider if you are trying to make the
decision of whether to use SaaS: Consider the functionality gap that still
exists between the SaaS solution and traditional locally hosted solutions,
security needs as your data is often stored across multiple servers with many
other clients data, and whether their will be consideration in the future of an
in house installation.

I plan on using this article to help analyze the applicability of Cloud
Computing as seen from a business perspective. While my topic is not directly
related to business applicability, this article does mention the ability to
reduce costs through savings in infrastructure by outsourcing that computing
power, which should be useful in analyzing the computing needs of clients
accessing cloud services

Bleicher, P. (2004). How Do I Buy Software? Let Me Count the Ways. Applied Clinical Trials, 13(6), 34-38.

The author enumerates many of the different ways that software can be be bought
today in the contexts of both the enterprise and the individual, ranging from
shrink-wrapped, perpetual or term licensing to one-time use software (for
instance TurboTax), to the new on-demand model (the cloud model) in the context
of attempting to calculate the total cost of ownership.

The author has done a decent job of briefly covering different software
ownership models and does enumerate some of the savings associated with going
with an on-demand model for their software solutions. Again, this is only
minimally useful as a technical reference for what client-hardware
configurations need to be, but at the same time it does provide an interesting
business perspective on Cloud Computing and on the other models of business
software purchasing.

Hayes, B. (2008). Cloud Computing. Communications of the ACM, 51(7), 9-11.

In this article, the author attempts to document the current trends in Cloud
Computing by first describing what it is, a movement of the locus of computation
from the desktop to the 'compute cloud', or the Internet. The author moves on to
briefly analyze it's similarity to the Time-Sharing model of ~50 years ago which
was the successor to the mainframe model that had been the norm since the
invention of the computer; a hub-and-spoke model which bears remarkable
similarity to the Cloud Computing model. Finally, the author notes 4 specific
trends (Cloud Office Suites, Enterprise Systems, Extreme Data Infrastructure,
Cloud OSes which focus on delivering the familiar OS environment via the
Internet instead of just through the browser) in an effort to help predict
possible directions that Cloud Computing might go. The author wraps up with a
discussion of some of the major challenges facing efforts to encourage Cloud
Computing: Scalability, the need to master multiple technologies, business
models, and ensuring privacy and security.

I would like to use this article to help make the connection between the
Time-Sharing, Computing-as-utility model that preceded the PC and today's Cloud
Computing, as well as to help define Cloud Computing. Beyond this, the article
really only provides conjectures and is not concerned with hard-proofs of any
given concept, just in the reality of the shift to Cloud Computing that is
happening currently. It does not cite many sources, nor does it appear to be
aimed at academia. However, it is from a reputable, peer-reviewed journal and so
should be dependable enough.

Lohr, Steve (2007). Google and I.B.M. Join in ‘Cloud Computing’ Research. NY Times. Retrieved 19 September, 2008, from http://www.nytimes.com/2007/10/08/technology/08cloud.html?_r=1&ex=1349582400&en=92a8c77c354521ba&ei=5088&partner=rssnyt&emc=rss&oref=slogin

Lohr's article on the new initiative being taken up by Google and IBM in late
2007 is an interesting overview of Cloud Computing in an academic environment.
It does not provide scholarly research into the topic and thus could not be used
in any authoritative way in the final research paper.  However, it does provide
interesting commentary from academics about how Cloud Computing is shaping up in
the future.

McDougall, P. (2007). Google Targets Microsoft With Launch Of Business Applications. Retrieved 13 January, 2009, from http://www.informationweek.com/news/internet/showArticle.jhtml?articleID=197007903.

This article covers the unveiling of the Google App Suite for Business (the
'Premiere' edition).  It mentions that the public version was funded by ads
while the business model here will be ad free but subscription based.

Peterson, Robyn (2008). What You Need To Know About Cloud Computing. PC Magazine. Retrieved 19 September, 2008, from http://www.pcmag.com/print_article2/0,1217,a%253D231900,00.asp

This article provides a great overview of Cloud Computing and its current use in
industry.  However, since it is not peer reviewed it will not be a great source
alone in my final research paper.  However, with other supporting content, it
should be nice to have.

Reiss, S. (2008). Cloud Computing. Available at Amazon.com Today. Wired Magazine, 16.05. Retrieved November 4, 2008, from http://www.wired.com/techbiz/it/magazine/16-05/mf_amazon?currentPage=all

This article covers the evolution of Amazon's cloud computing service, Amazon
Web Services.  It goes through the business model being used, a brief overview
of the history of computing as a utility, and the different companies that are
beginning to compete in the market like IBM, Google, and Microsoft.

Yahoo (2009). Cloud Computing. Retrieved 12 January, 2009, from http://research.yahoo.com/Cloud_Computing.

Yahoo's page documenting their Cloud Computing offerings.

\section{Netbooks, Smart-Phones}

AP. (2009). Netbooks power down for portability. Retrieved 20 January, 2009, from http://www.australianit.news.com.au/story/0,24897,24932480-5013037,00.html.

This article offers a brief overview of the current offerings in the Netbook
space, concluding that for at least the near future, Netbooks will be a
significant player in the mobile computing market.

Bergevin, P. (2008). Thoughts on Netbooks. Retreived 3 March, 2009, from http://blogs.intel.com/technology/2008/03/thoughts_on_netbooks.php.

This is a blog entry by an executive at Intel that covers briefly what Netbooks
are, why they are currently successful, and potential markets that they may be
able to serve.

Best, J. (2006). Analysis: What is a smart phone?. Retrieved 14 January, 2009, from http://networks.silicon.com/mobile/0,39024665,39156391,00.htm.

This article is an extensive discussion of what smartphones are.

Copeland, M. (2008). Disruptors: The 'netbook' revolution. Fortune Magazine. Retrieved 15 January, 2009, from http://money.cnn.com/2008/10/13/technology/copeland_asus.fortune/index.htm.

This article is a brief retrospective on the Eee PC from ASUS in the context of
today's burgeoning netbook market.  It talks briefly about how they are designed
primarily to work with today's Web 2.0 software and how many larger companies
such as Dell and HP are beginning to offer their own due to the smaller
companies' successes.
\end{document}
